\documentclass[final]{IEEEphot}

\usepackage{Socialst}
\renewcommand{\baselinestretch}{1.2}
\jvol{xx}
\jnum{xx}
\jmonth{May}
\pubyear{2018}

\begin{document}

\title{General Relativity \\
\VS
\NM{by precise approach}}

\author{Axect}

\affil{\textbf{M.S.}: Dept of Physics, Yonsei University \\
\textbf{B.S.}: Dept of Astronomy, Yonsei University }

\maketitle

\tableofcontents

\newpage

\section[Preliminaries]{Preliminaries}

\subsection{Manifolds}

\vs

\begin{tcolorbox}[colback=white!5!white,colframe=white!50!black, title=\textbf{Def 1.1 } Topological Manifolds]
  A \textit{manifold} $M$ of dimension $n$ is a topological space with the following properties.
  \begin{enumerate}
    \item $M$ is Hausdorff
    \item $M$ is locally Euclidean of dimension $n$
    \item $M$ has a countable basis of open sets
  \end{enumerate}
\end{tcolorbox}

\textbf{Why?}

\begin{itemize}
  \item \textbf{Hausdorff} : In Hausdorff space, convergent sequences converge to only one point.
  If you want to do calculus, you should need Hausdorff space.
  \item \textbf{Locally Euclidean} : This is the main reason that why we require manifolds.
  \item \textbf{Countable Basis} : We need \textit{partition of unity} to bring many properties of Euclidean space.
  For Hausdorff space, existence of partition of unity require \textit{paracompactness}.
  And paracompactness follows from \textit{second countability}. It is same as have countable basis.
\end{itemize}

%Property Template :
%
%\begin{property}[Simple function with Indicator function]
%	\label{prop:1}
%	If $\varphi:\chi \rightarrow \R$ is simple function then we can write
%	$$\varphi = \sum_{i=1}^{n}a_iI_{E_i}$$
%	where $\{a_i\}$ is image of $\varphi$ and $E_i = \varphi^{-1}(\{a_i\})$.
%
%	\HL
%\end{property}
%
%Example Template :
%
%\begin{example}[Example of Lebesgue Integral for simple function]
%	$f$ is a simple function given as:
%
%	$$ f =
%	\begin{cases}
%		1 & \text{if} 0 \leq x < 1 \\
%		2 & \text{if} 1 \leq x < 3 \\
%		3 & \text{if} 3 \leq x < 4 \\
%		0 & \text{otherwise}
%	\end{cases}
%	$$
%
%	then find integral of $f$ in $\R$.
%
%	\HL
%\end{example}
%
%Figure Template :
%
%\begin{figure}[h!]
%	\centering
%	\begin{subfigure}{.6\textwidth}
%		\centering
%		\includegraphics[width=\textwidth, height=9.5cm]{simple.png}
%	\end{subfigure}
%	\caption{Simple Function Example}
%\end{figure}
%
%Theorem Template :
%
%\begin{theorem}[Beppo - Levy Theorem]
%	If $f_n \rightarrow f~(mod~\mu)$ in a monotone increasing way then
%	$$\int \lim_{n\rightarrow\infty}f_n d\mu = \lim_{n\rightarrow\infty}\int f_n d\mu$$
%
%	\HL
%\end{theorem}
%
%\begin{theorem}[Fatou's Lemma]
%	Let $(S, \F, \mu)$ be a measure space and $\forall n \in \mathbb{N}$, $f_n : S \rightarrow [0,\infty)$ be measurable function. Then
%	$$\int_S \liminf_{n\rightarrow\infty}f_n d\mu \leq \liminf_{n\rightarrow\infty} \int_S f_n d\mu$$
%
%	\HL
%\end{theorem}


%% \ackrule

%\bibliographystyle{IEEEtran}
%\bibliography{thesis}

%\section*{Biographies}

%\textbf{P. W. Wachulak} received the degree${\ldots}$ \\[6pt]
%\textbf{M. C. Marconi} received the degree${\ldots}$ \\[6pt]
%\textbf{R. A. Bartels} received the degree${\ldots}$ \\[6pt]
%\textbf{C. S. Menoni} received the degree${\ldots}$ \\[6pt]
%\textbf{J. J. Rocca} received the degree${\ldots}$


\end{document}
